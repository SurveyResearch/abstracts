\documentclass[11pt,twoside]{article}
\usepackage[toc,page,header]{appendix}
\usepackage{pdfpages}
\usepackage{csquotes}
\usepackage{changepage}
\usepackage{fontspec}
\defaultfontfeatures{Scale=MatchLowercase}
\setmainfont[Mapping=tex-text]{TeX Gyre Termes}
\setsansfont[Mapping=tex-text]{TeX Gyre Heros}
\setmonofont{Courier}

\usepackage{float}
\usepackage{turnstile}
\usepackage{bussproofs}

\usepackage{geometry}
\geometry{letterpaper}

\newtheorem{theorem}{Theorem}
%\newtheorem{cor}{Corollary}
%\newtheorem{lem}{Lemma}
%\theoremstyle{remark}
\newtheorem{remark}{Remark}

\newtheorem{objection}{Objection}
\newenvironment*{response}[1][]{\noindent
\textbf{Response to Objection #1.}
\begin{adjustwidth}{1em}{1em}
}
{\end{adjustwidth}
\vspace{1ex}
}


%\usepackage[parfill]{parskip}    % Activate to begin paragraphs with an empty line rather than an indent

\usepackage{graphicx}
\usepackage[leftcaption]{sidecap}
\sidecaptionvpos{figure}{c}

%\usepackage{amssymb}

\usepackage{epstopdf}
\DeclareGraphicsRule{.tif}{png}{.png}{`convert #1 `dirname #1`/`basename #1 .tif`.png}

\usepackage[
bibstyle=numeric,
citestyle=authortitle,
natbib=true,
hyperref,bibencoding=utf8,backref=true,backend=biber]{biblatex}

\usepackage{hyperref}
\hypersetup{
    bookmarks=true,         % show bookmarks bar?
    unicode=true,          % non-Latin characters in Acrobat’s bookmarks
    pdftoolbar=true,        % show Acrobat’s toolbar?
    pdfmenubar=true,        % show Acrobat’s menu?
    pdffitwindow=false,     % window fit to page when opened
    pdfstartview={FitH},    % fits the width of the page to the window
    pdftitle={Papers},    % title
    pdfauthor={Author},     % author
    pdfsubject={Subject},   % subject of the document
    pdfcreator={Creator},   % creator of the document
    pdfproducer={Producer}, % producer of the document
    pdfkeywords={keyword1} {key2} {key3}, % list of keywords
    pdfnewwindow=true,      % links in new window
    colorlinks=true,       % false: boxed links; true: colored links
    linkcolor=blue,          % color of internal links
    citecolor=blue,        % color of links to bibliography
    filecolor=magenta,      % color of file links
    urlcolor=cyan           % color of external links
}
\usepackage{draftwatermark}


\usepackage{fancyhdr}
\setlength{\headheight}{15.2pt}
\pagestyle{fancy}

\lhead[Abstracts]{\thepage}
\chead[]{}
\rhead[\thepage]{Abstracts}

\title{\SR{} Interviewing \\
\vspace{12pt}\Large{Planned Papers}}
\author{G. A. Reynolds}
\date{\today}
\bibliography{%
../bib/abstracts.bib,%
../bib/brandom.bib,%
../bib/causality.bib,%
../bib/coginterviewing.bib,%
../bib/em.bib,%
../bib/logic.bib,%
../bib/mind.bib,%
../bib/philosophy.bib,%
../bib/pragmatism.bib,%
../bib/psychomet.bib%
../bib/psychometrics.bib,%
../bib/misc.bib,%
../bib/measurement.bib,%
../bib/psychology.bib,%
../bib/variables.bib,%
../bib/val.bib,%
../bib/validity.bib,%
}

%% Macros

\newcommand{\SM}{Standard Model}
\newcommand{\XSM}{Extended Standard Model}

\newcommand{\SMeth}{Survey Methodology}

\newcommand{\SR}{Survey Research}
\newcommand{\sr}{survey research}
\newcommand{\SRIV}{Survey Interview}
\newcommand{\sriv}{survey interview}
\newcommand{\SIV}{Survey Interviewing}
\newcommand{\FI}{Field Interviewer}
\newcommand{\Iver}{Interviewer}
\newcommand{\R}{Respondent}
\newcommand{\LPR}{Legal Permanent Resident}
\newcommand{\ART}{Assimilated Response Technique}
\newcommand{\GAM}{Grouped Answer Method}
\newcommand{\IOM}{Instrument of Measurement}

\includeonly{%
%% pilots,cards
}
%%%%%%%%%%%%%%%%%%%%%%%%%%%%%%%%%%%%%%%%%%%%%%%%%%%%%%%%%%%%%%%%
\begin{document}
\maketitle
\nocite{*}

\begin{abstract}
This document contains (draft) abstracts of some papers I'm working
on.  It's incomplete and rough, but readable.  The topics cover a lot
of ground, and the papers are pretty ambitious, so some of the
abstracts will require substantial revision and refinement, but I
think the basic ideas should be clear enough.

Drafts of some of the papers (in various states of
completion/disorder), as well as the source for this document, can be
found at \url{https://github.com/SurveyResearch}.

The general background for all of these papers is: 1) survey research
strikes me as extraordinarily parochial, and 2) the past several
decades have seen a great deal of ferment in the human (including
social) sciences, to the point that some argue that we are in the
middle of a real sea change in our thinking about the sciences,
philosophy, and their relations to each other and to human phenomena.
In particular, pragmatist philosophers (Rorty, Putnam, Brandom, Price,
etc.) have elaborated compelling and detailed alternatives to many of
the concepts that have dominated Western thought since at least the
17th century, and cognitive science has moved far beyond the naive
computationalist models that dominated its early ``classical'' phase.
So the overall theme of these papers is an attempt to rethink survey
research in light of these developments.
\end{abstract}

\tableofcontents
%% \listoffigures

\newpage
%%%%%%%%%%%%%%%%%%%%
\section{Pragmatism and \SR{}}

\begin{abstract}
\end{abstract}

%%%%%%%%%%%%%%%%%%%%
\newpage
\section{Mensuration without Representation}

\begin{abstract}

  There are two basic problems with measurement in the social
  sciences.  One is that the relentless drive to emulate physics as
  the model science has lead to an overly-narrow focus on metric
  structure at the expense of algebraic (and other) mathematical
  structure.  The other is that measurement has been construed almost
  exlusively in positivistic, representationalist (etc.) terms.  But
  representation is not essential to measurement; we can discard it
  without compromising effective measurement.  A pragmatic conception
  of measurement has no need of the concept of representation, but
  works just as well.  But pragmatism about measurement is not just
  another option; rather, we intend to show that the pragmatist
  perspective provides the most compelling account of measurement as
  actually practiced (and conceptualized) in the sciences.

  This paper is organized as follows.  It begins with an overview of
  the major themes of contemporary pragmatism, with particular focus
  on inferential semantics and linguistic expressivism.  Conceptual
  content is viewed as inferentially articulated, as opposed to
  representatial.  The inferential structure is instituted by
  proprieties of practice.  The representational dimension of language
  use is to be explained in terms of the social structure of the
  discursive practices that institute meaning.  The role of language
  is to express rather than represent; it allows us to say what we can
  otherwise only do.  Pragmatism turns away from metaphysical
  questions like ``What \textit{is} measurement, \textit{really}?'' in
  favor of questions like ``What role does measurement play in our
  lives?  How is it used?  What counts as \textit{doing} it?''

We then lay the groundwork for a revised concept of measurement by
examining mathematical foundations.  We examine basic concepts such as
number, magnitude, counting, ordering, etc. with special attention to
practice; for example, we characterize counting in terms of what one
must be able to \textit{do} in order to count as counting (put things
into one-to-one correspondence).  We then provide a brief overview of
algebraic concepts, and show that the mathematics of measurement can
be viewed as primarily a matter of algebraic rather than metric
structure.  From this perspective, metric measurement comes out as a
species of a more general notion of mathematicization.

  Then we move to an overview of the major theories of measurement on
  the contemporary scene, especially in the human (behavioral)
  sciences, with special focus on the role of measurement in \SR{}.

Next, we debunk the Myth of Measurement Levels, which plays such a key
role in social scientific research.  Steven's famous model of four
``levels'' of measurement construes the four scales as elements of a
hierarchical structure in which each ``level'' (scale) subsumes those
preceding it.  But it does not fit the (mathematical) facts.  For
example, mathematically, the concept of an order (ordinal scale) does
not presuppose the concept of a countable set (nominal scale).
Furthermore, the concepts of line (infinite in both ``directions'')
and ray (infinite in one direction) are distinct; a ray is not a kind
of line, nor vice-versa, so ratio scales (rays) do not subsume
interval scales (lines) -- nor vice-versa.  Historically and
conceptual, the notion of a ray (``line'' with absolute origin)
precedes the notion of a line (``line'' with arbitrary center), so if
there is a hierarchy of levels here, it is the reverse of Stevens'
hierarchy: ratio scale precedes interval scale.  The way to remedy
this situation is to recognize that what matters is not quantitative
(mathematically: metric) measurement, but algebraic structure.

But Stevens' model is doubly pernicious: not only is it beset by
mathematical confusion; more damaging is its narrowness of vision.  By
assuming that \textit{quantitative} measurement is the name of the
game, Stevens' model excludes entire classes of mathematical structure
from consideration.  But many non-metric mathematical structures play
key roles in the natural sciences; for example, chemistry relies on
Group Theory to describe symmetries.  Whether symmetries
characterizable in terms of Group Theory (or other algebraic
structures) are to be found in social and psychological phenomena is
an empirical question, but Stevens' model excludes the possibility
from the beginning.  Or more accurately, it places the possibility
outside of the researcher's field of vision.

This leads to the notion that empirical measurement should be viewed
in terms of assigning \textit{algebraic} structures to empirical
systems; this is a broader notion that the classic idea of applying
\textit{metric} structures to such systems.

  Having examined the theories, we step back and address the more
  general issue of critera of adequacy for any theory of measurement.
  Any account of measurement must address the three fundamental
  aspects of measurement: mathematical vocabulary, empirical
  vocabulary, and their relation to each other and to the world.  In
  other words, measurement always involves at least two vocabularies:
  a vocabulary of mathematics and an empirical vocabulary, and the
  task of the theory is to align them and make the latter ``match''
  the world.  This section of the paper examines the pragmatic
  dimensions of these aspects: what features must be exhibited by
  practices using these vocabularies in order for those practices to
  count as measurement practices?

  Then we proceed to the critical part of the paper.  I show how the
  pragmatist perspective exposes problems in the popular accounts of
  measurement, with special focus on the survey research.  In
  particular, I show that some of the most basic measurement-related
  doctrines of orthodox survey research do not answer to the facts of
  the matter.  For example, I show that the idea that a question is an
  instrument of measurement, and that asking a question and recording
  an answer measures something, is based on deep confusion about the
  nature of measurement and discursive practice.

  Finally we move to the constructive part of the paper.  I show how
  an acceptable account of measurement can be constructed out of
  purely pragmatist materials, and how survey interviewing can be used
  to produce scientifically useful information even without the
  positivistic models that has dominated it throughout its history..
\end{abstract}

%%%%%%%%%%%%%%%%%%%%
\newpage
\section{Deflating Validity}

\begin{abstract}
In philosophy and logic, validity and truth are closely related.
Truth is a property of sentences (propositions); validity is a
property of inferences.  In recent decades, ``deflationary'' (or
``minimalist'') accounts of truth have become increasingly popular
among philosophers.  Broadly speaking, these accounts deny that truth
is a substantial property, and instead treat the term ``truth'' as a
kind of expressive device; it adds nothing significant to the
expressions in which is appears, but it makes the language
significantly more powerful.  It allows us to say things we
otherwise could not say, or could only say in cumbersome ways.  For
example, with a locution like ``... is true'', we can endorse claims
by naming them (e.g. ``Fermat's last theorem is true''); without such
a locution, we would have to explicitly repeat the theorem as a claim
(e.g. ``There is no integer z greater than 2 such that ...'').  And
some things we can say with ``... is true'' would be practically
impossible to express without it, such as ``everything the policeman
said is true'' (since it would not be possible to repeat everything he
said) or ``the theorems of group theory are true'' (since there are (I
assume) infinitely many such theorems).

A third aspect: sentences contain referring components.  To the truth
of a sentence corresponds the ``referentiality'' of its components.
``Snow is white'' is true; it is true because ``Snow'' refers to the
famous cold stuff, and ``white'' refers to the famous color.  We need
(but generally speaking do not have) a technical term to refer to the
property of such refering relations that corresponds to the property
of truth of sentences.  It is a category mistake to say ```Snow' is
true'', but we would like to say ```Snow' is $x$'' in order to bring
attention to this truth-like referential condition.  In \SR{} (and
social science in general), the term ``validity'' is often recruited
to serve this need in measurement vocabulary.  The inadvisability of
this becomes obvious when you move from measurement to description:
``2.3 meters is a valid measurement of the length of x'' is a common
way to talk, but ```Snow' is valid'' sounds decidedly off-key.

This paper has two goals.  The theoretical goal is to do with validity
what deflationists have done with truth.  The more practical goal is to
examine the use and role of the concept (term) validity in \SR{}.

The first part of the paper thus explores the plausibility of a
deflationary or minimalist concept of validity.  Not just logical
(inferential) validity, but validity as used by the social sciences,
as a property of referential relations.

The second part of the paper examines the notion of validity as used
in \SR{}.  Suffice it to say that vocabulary of validity in the social
sciences, especially psychology and education research, is very,
\textit{very} confused.  Generally speaking, the term is used to
refer, not to inferences and their properties, but to referential
relations.  Classic definitions of validity in the social sciences
usually say something like ``measures what it purports to measure'',
which is to say, measurement expressions (e.g. ``2.3 meters'')
\textit{refer} to entities (properties, relations) in the world.  But
it is also used to refer to inferences and a variety of other
concepts.



The connection between the first and second parts is that the social
sciences usually treat validity as a substantial property.  Theories
of validity often take on a metaphysical hue; they attempt to say what
validity \textit{is}, as if it were some kind of entity or substance
-- validity stuff -- that referring terms ``have'', possibly in
greater or lesser degrees.  On the deflationary view, this is a
mistake that inevitably leads to unresolvable problems.

\end{abstract}

%%%%%%%%%%%%%%%%%%%%
\newpage
\section{Why `True 'Values' Are Not Important in Survey Research}

\begin{abstract}
See \cite{brandom_why_2009}
\end{abstract}

%%%%%%%%%%%%%%%%%%%%%%%%
\newpage
\section{Reliability}

\begin{remark}
  This abstract needs a lot of work.
\end{remark}

\begin{abstract}
This essay argues against the use of statistical concepts of
reliability in \SR{}.  Such concepts only describe the past - data
already gathered.  But the notion of reliability essentially involves
present and future; to call something ``reliable'' is implicitly to
make a prediction about the future.

So the critical question is how we can make decisions about the
reliability of instruments, procedures, practices, etc.  Statistical
analysis has a role to play here, but cannot decide the issue;
statistical measures of variance in past observations cannot by
themselves say anything about the likelihood of reliability of future
observations.

Reliability judgments are about the future.  We want to know ``Can I
rely on this the next time I use it?''

The Standard Model of survey research seeks to show that questions are
reliable; that is, that a given question can be relied on in future
uses.  More specifically, a question is reliable to the extent that,
when administered properly (usually this means use of ``Standardized
Interviewing'' methods), it \textit{will} measure what it purports to
measure, so that it yields good, ``comparable'' data.  The standard
means of establishing this sort of reliability involves statistical
analysis of \textit{past} performance; lower variance means higher
reliability [FIXME?].  But this is not enough; what is missing is a
theory that links past to future.

Compare temperature measurement.  Here too statistical analysis is
used to provide evidence of reliability, but predicatability is only
available by virtue of a theory of heat that explains temperature
measurement.  That is what provides the basis of projecting future
from past performance.  Generally speaking, this sort of theoretical
basis goes missing in survey research reliability studies.

But that is not all.  All measurement involves a viscious circle.  In
the case of temperature measurement, a good theory of heat is
necessary but not sufficient to prove that temperature is in fact
quantitatively measureable.  Such a theory only provides
presuppositions.

\begin{remark}
  TODO: summary of how the circle works in temperature measurement
  (\cite{chang_inventing_2004}, \cite{sherry_thermoscopes_2011}).
\end{remark}

What's missing is an account of the essentially pragmatic nature of
measurement.  The way we arrive at an acceptable notion of temperature
measurement is by repeated cycles of hypothesis-test-revise, not by
deductive proof.  This cycle never yields proof or truth; the best it
can deliver is usefulness (etc.).  The reason we think temperature is
quantitatively measureable is because we have managed to create
\textit{effective} concepts and measurement procedures - that is,
concepts and methods that have proved successful in describing and
manipulating the world -- and that have continued to improve.
\textit{Proof} of a theory of quantifiable temperature has always been
and will always remain beyond our grasp.  Such a proof would require a
variety of questionable ontological commitments that we simply do not
need.

The second part of this paper is constructive.  It attempts to
construct an alternative notion of reliability in survey research that
is anchored in acknowledgment of the essentially pragmatic and
open-ended nature of measurement.  It proposes that statistical
analyses of past performance be complemented by a concept of
reliability as an intentionally constructed feature of survey
interviews.  The basic idea is analogous to concepts of quality
assurance used in manufacturing: reliability (and other quality
attributes) are viewed as something that can be guaranteed by
construction, rather than merely measured after the fact.  In
manufacturing, this translates into efforts to identify and remove
causes of (and opportunities for) defects in the production process.
In survey interviewing, this translates into efforts to structure the
questionnaire and the interview such that the respondent's grasp of
the meaning of questions is actively constructed, rather than left to
chance.

To a large extent this is a matter of reconceptualizing the survey
interview.  It moves away from the standard ``laboratory model'' of
survey research, in favor of a collaborative model that recognizes the
fundamentally social and constructive character of discursive
practice.  Question-answer sequences in an interview are
\textit{always} constructed by the participants, but the standard
model pretends otherwise.  The approach suggested here merely
recommends that researchers acknowledge the constructive,
collaborative character of interviewing and use that knowledge to
acheive their goals.

\end{abstract}

%%%%%%%%%%%%%%%%%%%%%%%%
\newpage
\section{Error}

\begin{abstract}
A pragmatic perspective on survey interviewing requires that we rethink the concept of ``survey error''.
\end{abstract}

%%%%%%%%%%%%%%%%%%%%%%%%
\newpage
\section{Speech, Discourse, Language: a survey of contemporary models and their relevance to \SR{}}

\begin{abstract}
A common lament in the \SR{} literature is the lack of a good model of
what is variously referred to as the ``survey interview process'', the
``question-answer process'', the ``response process'', or the like.
Where researchers do articulate an explicit model, they tend to rely
on the sort of cognitivist model exemplified by \cite{tourangeau_psr}.

But today we have a great variety of distinct models of speech,
discourse, and language.  The purpose of this paper is to critically
survey some of the best known such models and examine their relevance
to survey research interviewing.

The first part of this paper provides background.  It begins with a
brief overview of the critical distinction between the natural ``space
of laws'' and the cultural ``space of reasons''.  Discourse is
obviously dependent on the causal realm; it's hard to talk without a
body, or to imagine a mind without a brain.  Yet the
\textit{intelligibility} of discursive behaviour (speech, language)
seems to call for a distinctive order of explanation, one that swings
free of cause and effect and instead appeals to notions of normativity
and rationality.  This first section examines the tension between
these orders of explanation and provides a general overview of some of
the distinct (and sometimes incompatible) ways of addressing it.  It
concludes that any model adequate to the needs of survey research
should disregard the causal realm of the states and processes
underlying discursive practices, and instead focus on the rational
structure of those practices - what Wilfrid Sellars dubbed ``the space
of reasons''.

This section also provides a brief overview of the dominant modes of
linguistic thought in the 20th century, with particular attention to
Chomskyism.  The main purpose of the overview is therapeutic:
Chomskyism is basically dead, but, zombie-like, it refuses to die, and
many survey researchers accept it (or some variant) uncritically.
Many of its tenets (competence v. performance, language acquisition
v. language learning, Universal Grammar) are still defended by some
specialists (e.g. Pinker) and continue to enjoy uncritical acceptance
by non-specialists (including in particular survey research
methodologists).  So one purpose of this section is to expose the
problems with 20th century ``scientific'' linguistics in general and
Chomskyism in particular.

Finally, it sketches some of the main relevant themes from cognitive
science, neuroscience, and the philosophy of language.


The second part examines three (four?) distinctive approaches to the
study of discursive practice.

\textit{Ethnomethodology} and its offshoot \textit{Conversation
  Analysis} seek to understand such behavior in terms of the local,
accountable order actively produced by participants in discourse.  It
reverses the standard sociological order of explanation, which seeks
to understand the doings of individuals in terms of causal forces
exerted by social entities and processes.  CA instead examines the
fine detail of actual situated episodes of discursive behavior in
order to discover how participants (``members'') manage to produce and
sustain discourse as a local, situated phenomenon.

\textit{Dialogism} the rubric adopted by a number of scholars (mainly
northern Europeans in psychology departments) for a framework or
collection of doctrines traceable (mainly) to Mikhail Bakhtin but also
indebted to e.g. G. H. Mead.  It is largely motivated by Bakhtin's
observation that \textit{utterance} is essentially dialogical; it
always presupposes not only a speaker, but also \textit{responsivity}
and \textit{addressivity}.  This approach categorically rejects the
atomistic, monological perspective that usually characterizes
cognitivist approaches.  To understand discursive episodes, one must
understand a complex whole in which the parts (individual utterances)
are always essentially interrelated.

\textit{Integrationism} is an approach to linguistics advocated by the
linguist Roy Harris in reaction to the sort of structuralist,
cognitivist conceptions of linguistics that have dominated the field
since the days of de Saussure.  In particular, it rejects what Harris
calls the ``telementation model'' of language, according to which
discourse is reducible to the encoding, transmission, and decoding of
thought.  Although this is primarily a model of linguistics, it is
closely related to dialogical models and has direct relevance to the
study of discursive practices.

\textit{Pragmatism} begins by asking what counts as discursive
\textit{practice}.  Instead of asking ``what is it?'', it asks
questions like ``what role does it play in our lives?'' and ``what
must one \textit{do} in order to count as deploying it?'', etc. (Huw
Price, et al.)  The most thoroughly worked out pragmatist model of
discursive practice is the \textit{deontic scorekeeping} model of the
philosopher Robert Brandom.  This is a very sophisticated account of
the pragmatic foundations of discursive (and thus conceptual)
practice, elaborated by Brandom in his 1994 masterpiece
\citetitle{brandom_mie} and many other works.

One striking fact emerges: these approaches may employ distinctive
vocabularies, and may be incompatible in various ways, yet they are
all clearly taking similar approaches to more-or-less the same sorts
of things.  They recognizably involve variants on a few master
concepts: the primacy of practice (and hence of empirical
investigation over speculative theorizing); the situated or
context-dependent nature of meaning; the essentially social nature of
language, thought, and communication; etc.  (Another way to put this
might be to say that they share a common enemy, one that appears in a
variety of guises: monologism, cartesianism, atomism,
representationism, etc.)

The final section of the paper explores the relevance of such models
(which I group under the general rubric of ``pragmatism'') for \SR{}.

\end{abstract}

%%%%%%%%%%%%%%%%%%%%
\newpage
\section{The Conduct of the Survey Interview: Models and Protocols}

\begin{abstract}
This paper analyzes and compares three models of survey
interviewing.  That is, models of the conduct of survey interviewing,
rather than models of the structure of questionnaires, interviews,
etc.  

The first is the Laboratory Model, which is motivated by a desire to
mimic the experimental physical sciences, paradigmatically physics.
The paradigmatic example of this sort of model is the ``Standardized
Survey Interview''.  Analysis of this model exposes a variety of
(usually) unacknowledged commitments to theoretical/philosophical
doctrines, which are shown to be untenable.

The second model is the Extended Laboratory Model.  This is a
modification of the Laboratory Model.  It acknowledges that, due to
the interactive nature of the interview, the interviewer inevitably
makes a contribution.  But it retains the basic structural commitments
of the laboratory Model.  An example of an Extented Laboratory model
is Maynard et al's ``alternating model''.


The third model is The Theatrical Model.  This model is similar to the
Laboratory Model, in that it recommends that the interviewer read the
questions exactly as written, avoid probes, etc., but it involves a
very different conceptualization of the nature of interviewing.  Like
the Extended Laboratory Model, it acknowledges that the Field
Interviewer makes a substantial contribution to the survey interview,
due to the fundamentally interactive and collaborative nature of
discursive practice.  But it stresses that interviewing essentially
involves role-playing.  This may or may not result in interviewer
behavior that is different from what it would be under the Extended
Laboratory Model, but either way it would suggest different approaches
to interviewer training.  This model is based on a more realistic
picture of the nature of surveys and survey interviewing as
traditionally practiced, but it also has some weaknesses, which we
analyze.

Finally, the fourth model is The Collaborative Model.  This model is
driven by a closer and more realistic analysis of the nature of the
survey interview.  It demystifies aspects of the interview that the
other two models take for granted or ignore, such as the various
asymmetries involved in interviews, the fact that completion of a
survey questionnaire is the joint responsibility of the interviewer
and the respondent, and so forth.  It discards the fictions that are
at the core of the other models discussed.  Most critically, motivated
by considerations of the nature of discursive practice and the
production of meaning, it denies that survey interviewing involves
measurement.  In summary, this model recommends that survey
interviewing be construed as collaborative or joint action, and that
the demystified facts of the matter be openly acknowledged in the
conduct of interviews.  This means, among other things, that the field
interviewer should serve as an assistant to the respondent, rather
than a proxy for the researcher; that interviewer and respondent are
jointly responsible for completing the questionnaire; and that the
results of individual survey interviews should be viewed as a trace a
kind of dialog between the individuality of the particular respondent
and the stereotype presupposed by the questionnaire design.
\end{abstract}

%%%%%%%%%%%%%%%%%%%%
\newpage
\section{A Critique of the Theory of Cognitive Interviewing}

\begin{abstract}
  
\end{abstract}

%%%%%%%%%%%%%%%%%%%%%%%%
\newpage
\section{A Quality Assurance Model for \SR{}}

\begin{abstract}
abstract
\end{abstract}



\clearpage
\appendix
\begin{appendices}
\section{Bibliography}
%% \addcontentsline{toc}{chapter}{Bibliography}
%% \bibliographystyle{plainnat}
\printbibliography[heading=none]
\end{appendices}

\end{document}
