\documentclass[11pt,twoside]{article}
\usepackage[toc,page,header]{appendix}
\usepackage{pdfpages}
\usepackage{csquotes}
\usepackage{changepage}
\usepackage{fontspec}
\defaultfontfeatures{Scale=MatchLowercase}
\setmainfont[Mapping=tex-text]{Times New Roman}
\setsansfont[Mapping=tex-text]{Arial}
\setmonofont{Courier}

\usepackage{float}
\usepackage{turnstile}
\usepackage{bussproofs}

\usepackage{geometry}
\geometry{letterpaper}

\newtheorem{theorem}{Theorem}
%\newtheorem{cor}{Corollary}
%\newtheorem{lem}{Lemma}
%\theoremstyle{remark}
\newtheorem{remark}{Remark}

\newtheorem{objection}{Objection}
\newenvironment*{response}[1][]{\noindent
\textbf{Response to Objection #1.}
\begin{adjustwidth}{1em}{1em}
}
{\end{adjustwidth}
\vspace{1ex}
}


%\usepackage[parfill]{parskip}    % Activate to begin paragraphs with an empty line rather than an indent

\usepackage{graphicx}
\usepackage[leftcaption]{sidecap}
\sidecaptionvpos{figure}{c}

%\usepackage{amssymb}

\usepackage{epstopdf}
\DeclareGraphicsRule{.tif}{png}{.png}{`convert #1 `dirname #1`/`basename #1 .tif`.png}

\usepackage[
bibstyle=numeric,
citestyle=authortitle,
natbib=true,
hyperref,bibencoding=utf8,backref=true,backend=biber]{biblatex}

\usepackage{hyperref}
\hypersetup{
    bookmarks=true,         % show bookmarks bar?
    unicode=true,          % non-Latin characters in Acrobat’s bookmarks
    pdftoolbar=true,        % show Acrobat’s toolbar?
    pdfmenubar=true,        % show Acrobat’s menu?
    pdffitwindow=false,     % window fit to page when opened
    pdfstartview={FitH},    % fits the width of the page to the window
    pdftitle={Papers},    % title
    pdfauthor={Author},     % author
    pdfsubject={Subject},   % subject of the document
    pdfcreator={Creator},   % creator of the document
    pdfproducer={Producer}, % producer of the document
    pdfkeywords={keyword1} {key2} {key3}, % list of keywords
    pdfnewwindow=true,      % links in new window
    colorlinks=true,       % false: boxed links; true: colored links
    linkcolor=blue,          % color of internal links
    citecolor=blue,        % color of links to bibliography
    filecolor=magenta,      % color of file links
    urlcolor=cyan           % color of external links
}
\usepackage{draftwatermark}


\usepackage{fancyhdr}
\setlength{\headheight}{15.2pt}
\pagestyle{fancy}

\lhead[Abstracts]{\thepage}
\chead[]{}
\rhead[\thepage]{Abstracts}

\title{\SR{} Interviewing \\
\vspace{12pt}\Large{Planned Papers}}
\author{G. A. Reynolds}
\date{\today}
\bibliography{%
../bib/abstracts.bib,%
../bib/brandom.bib,%
../bib/causality.bib,%
../bib/em.bib,%
../bib/logic.bib,%
../bib/mind.bib,%
../bib/philosophy.bib,%
../bib/pragmatism.bib,%
../bib/psychomet.bib%
../bib/psychometrics.bib,%
../bib/misc.bib,%
../bib/measurement.bib,%
../bib/psychology.bib,%
../bib/variables.bib,%
../bib/val.bib,%
../bib/validity.bib,%
}

%% Macros

\newcommand{\SM}{Standard Model}
\newcommand{\XSM}{Extended Standard Model}

\newcommand{\SMeth}{Survey Methodology}

\newcommand{\SR}{Survey Research}
\newcommand{\sr}{survey research}
\newcommand{\SRIV}{Survey Interview}
\newcommand{\sriv}{survey interview}
\newcommand{\SIV}{Survey Interviewing}
\newcommand{\FI}{Field Interviewer}
\newcommand{\Iver}{Interviewer}
\newcommand{\R}{Respondent}
\newcommand{\LPR}{Legal Permanent Resident}
\newcommand{\ART}{Assimilated Response Technique}
\newcommand{\GAM}{Grouped Answer Method}
\newcommand{\IOM}{Instrument of Measurement}

\includeonly{%
%% pilots,cards
}
%%%%%%%%%%%%%%%%%%%%%%%%%%%%%%%%%%%%%%%%%%%%%%%%%%%%%%%%%%%%%%%%
\begin{document}
\maketitle
\nocite{*}

\begin{abstract}
abstract
\end{abstract}

\tableofcontents
\listoffigures

\newpage
%%%%%%%%%%%%%%%%%%%%
\section{Pragmatism and \SR{}}

\begin{abstract}
\end{abstract}


%%%%%%%%%%%%%%%%%%%%
\section{Why `True 'Values' Are Not Important in Survey Research}

\begin{abstract}
See \cite{brandom_why_2009}
\end{abstract}

%%%%%%%%%%%%%%%%%%%%
\section{Mensuration without Representation}

\begin{abstract}
Measurement pragmatism.  No representation needed.
\end{abstract}

%%%%%%%%%%%%%%%%%%%%
\section{Deflating Validity}

\begin{abstract}
Semantic and metaphysical deflationism works as well for validity as
it does for truth and reference.
\end{abstract}

\begin{remark}
  Deflationism seems to depend essentially on some form of
  expressivism.  Or maybe they amount to the same thing?
\end{remark}

\subsection{Validity, Reliability, Error}
\label{sub:Validity}

\begin{remark}
What is the point of worrying about validity?  Is it something in the
world that we are trying to discover?  Then we're trying to find ``the
right description of the world'' (Putnam).  Or is it a concept, so
that validity talk is about conceptual analysis and definition?

Or: we try to find the right description, and validity talk is part of
how we decide that we have found it.

\end{remark}

\begin{remark}
Why do psychometricians and the like worry so about validity?

Hypothesis: when they say ``validity'', what they're really interested
in is scientific legitimacy.  Effectively, to say that a test (etc.)
is valid is to say that it is in fact scientific.  Thats the practical
import of the concept of validity for them.

Unpack this.  Expose the assumptions and implications.
\end{remark}

\begin{remark}
  The problem with validity (quantifiability) is circularity.  If the
  task is to show that some property is quantitative, we have to do
  this without relying on quantitative vocabulary.  So for example, if
  we want to show that temperature is quantitative, we cannot use the
  concept of a unit of temperature to do so, because that presupposes
  just the outcome we are supposed to demonstrate.  This is similar to
  the problem we face in seeking to account for representational
  vocabulary in non-representational terms.

  quantifiability v. validity?  distinct problems, but the latter
  depends on the former?
\end{remark}

key concepts:

\begin{itemize}
\item validity treated as a special kind of property - of what?
\item constructs
\item (latent) variables
\item indicators
\end{itemize}

``validity'' as code for:

\begin{itemize}
\item legitimacy
\item vindication
\item credibility
\item proof (good premises + valid inference)
\end{itemize}

\begin{remark}
  On the idea that validity something (a property, etc.) that we look
  for in scientific theories in order to distinguish good ones from
  bad: see Putnam on fact/value distinction.  We use value judgments -
  simplicity, parsimony, etc. - in every aspect of science (thought),
  esp. in weeding out bad theories.  For there is no external or
  objective criterion of acceptability for theories to which we can
  appeal, nor is there any such citerion that does not involve value
  judgments.
\end{remark}

\begin{remark}
  So along with the fact/value distinction, and the analytic/synthetic
  distinction, the internal/external distinction also collapses?  Or
  do we just exclude the notion of external?  No; we need to retain
  the idea of an external world that is independent of us and to which
  some of our judgments are answerable.  We don't get to just make
  stuff up and call it true (correct) for at least some of our claims.
  There is no external absolute authority that can decide for us which
  theories are true, or rather which we should endorse, but that does
  not mean there is no external world that is authoritative for some
  of our sayings.  But isn't that trying to have it both ways?  How
  can our theories answer to the world if we cannot appeal to the
  world or some other external authority to sort them out?  See
  Brandom.

Related issue: what counts as evidence?  How do we decide?  What are
we doing when we decide that something counts as strong (weak)
evidence in support of a theory?  What are the criteria of adequacy
for an account of evidence?
\end{remark}

\subsection{RCT and Self-validation}

See Cartwright on RCT as self-validating.  This seems to mean that
RCTs are valid by construction.

This nicely parallels industrial QA notions of guaranteeing quality by
designing a production process that prevents defects.

What's the logic here?  Is self-validation really possible?  How can a
process validate itself - isn't the very idea inherently circular?  Or
rather, don't we land in a regress?  After all, if the idea is to
specify a process that yields validity, how do we know that that
process is itself valid?

\subsection{Deflation}

How can we get out of this mess?  One way is to deflate the notion of
validity, just deny that it is a substantive property.  When we claim
that a result is valid etc. what we are really saying is that we
endorse it, approve of it, etc.  It's an expressive device.  Compare
the semantic deflationist's idea that calling something true amounts
to endorsing or approving of it.

So if we discard the notion of validity (since it does no real work),
don't we find ourselves lacking something essential?  Well, we just
need a vocabulary that allows us to say explicitly the sorts of things
we find it useful to be able to express with respect to a study or qx
technique.  For example: credibility, utility, legitimacy,
vindication, justification, etc.

\begin{remark}
  The notion of validity seems to be connected to the problem of
  deciding which theories we should endorse.  What are the criteria of
  adequacy for any notion (or theory) of validity?  Or: what are the
  requirements that should be met by any purported explanation of
  validity?  Both particular cases and the general idea.  Tarski gives
  us something like this for logical validity; what about ``validity''
  as the term is used by psychometricians, test theorists, etc.?

Contrast: claims of validity for a case, v. explanation of what
validity is.


\end{remark}

The objection will no doubt be that we need some kind of standard,
which is just to say that we want to measure this something (validity,
credibility, whatever).  Implicit in all this is the notion that there
is some ``objective'' fact of the matter to which our
study/technique/etc. is ansswerable. A study is valid iff - what?  If
it meets some definite ``objective'' criteria.  Methodological
criteria, conditions of validity, etc.  In the psychometrics and
testing tradition this appeal to external authority is expressed as
something along the lines of ``measures what it purports to measure''.
Which is only meaningful insofar as a) there is actually something
there to measure, and b) it is in fact susceptibel to measurement.

And usually this is expressed in statistical terms.  But that dog
won't hunt either - you cannot get to validity via statistics.  All
you can do is measure central tendencies and variance - not enough to
establish validity, which is a substantive notion. (analysis
elsewhere).

To say that sth is valid is just to say that it is admirable
(Peirce?), or perhaps that it is virtuous, that it has the virtues we
prize.

%%%%%%%%%%%%%%%%%%%%%%%%
\section{Reliability}

\begin{abstract}

\end{abstract}

%%%%%%%%%%%%%%%%%%%%%%%%
\section{Error}

\begin{abstract}

\end{abstract}

%%%%%%%%%%%%%%%%%%%%%%%%
\section{The Deontic Scorekeeping Model of Discursive Practice and \SR{}}

\begin{abstract}
Why the deontic scorekeeping model is preferable to others, esp. the
cognitive model.
\end{abstract}

\begin{remark}
  It's a model of discursive, that is rational, practice.  Contrast
  this with most models on offer which tend to focus on subpersonal
  processes; hence the prevalence of talk about ``the survey
  process'', the ``response process'', etc.
\end{remark}

%%%%%%%%%%%%%%%%%%%%
\newpage
\section{Models of Survey Interviewing}

\begin{abstract}
This paper analyzes and compares three models of survey
interviewing.  That is, models of the conduct of survey interviewing,
rather than models of the structure of questionnaires, interviews,
etc.  

The first is the Laboratory Model, which is motivated by a desire to
mimic the experimental physical sciences, paradigmatically physics.
The paradigmatic example of this sort of model is the ``Standardized
Survey Interview''.  Analysis of this model exposes a variety of
(usually) unacknowledged commitments to theoretical/philosophical
doctrines, which are shown to be untenable.

The second model is the Extended Laboratory Model.  This is a
modification of the Laboratory Model.  It acknowledges that, due to
the interactive nature of the interview, the interviewer inevitably
makes a contribution.  But it retains the basic structural commitments
of the laboratory Model.  An example of an Extented Laboratory model
is Maynard et al's ``alternating model''.

The third model is The Theatrical Model.  This model is similar to the
Laboratory Model, in that it recommends that the interviewer read the
questions exactly as written, avoid probes, etc., but it involves a
very different conceptualization of the nature of interviewing.  Like
the Extended Laboratory Model, it acknowledges that the Field
Interviewer makes a substantial contribution to the survey interview,
due to the fundamentally interactive and collaborative nature of
discursive practice.  But it stresses that interviewing essentially
involves role-playing.  This model is based on a more realistic
picture of the nature of surveys and survey interviewing, but it also
has some weaknesses, which we analyze.

Finally, the fourth model is The Collaborative Model.  This model is
driven by a closer and more realistic analysis of the nature of the
survey interview.  It demystifies aspects of the interview that the
other two models take for granted or ignore, such as the various
asymmetries involved in interviews, the fact that completion of a
survey questionnaire is the joint responsibility of the interviewer
and the respondent, and so forth.  It discards the fictions that are
at the core of the other models discussed.  Most critically, motivated
by considerations of the nature of discursive practice and the
production of meaning, it denies that survey interviewing involves
measurement.  In summary, this model recommends that survey
interviewing be construed as collaborative or joint action, and that
the demystified facts of the matter be openly acknowledged in the
conduct of interviews.  This means, among other things, that the field
interviewer should serve as an assistant to the respondent, rather
than a proxy for the researcher; that interviewer and respondent are
jointly responsible for completing the questionnaire; and that the
results of individual survey interviews should be viewed as a trace a
kind of dialog between the individuality of the particular respondent
and the stereotype presupposed by the questionnaire design.
\end{abstract}

%%%%%%%%%%%%%%%%%%%%
\newpage
\section{A Critique of the Theory of Cognitive Interviewing}

\begin{abstract}
  
\end{abstract}

%%%%%%%%%%%%%%%%%%%%%%%%
\section{A Quality Assurance Model for \SR{}}

\begin{abstract}
abstract
\end{abstract}



\clearpage
\appendix
\begin{appendices}
\section{Bibliography}
%% \addcontentsline{toc}{chapter}{Bibliography}
%% \bibliographystyle{plainnat}
\printbibliography[heading=none]
\end{appendices}

\end{document}
